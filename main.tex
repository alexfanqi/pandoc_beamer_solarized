% Options for packages loaded elsewhere
\PassOptionsToPackage{unicode}{hyperref}
\PassOptionsToPackage{hyphens}{url}
\PassOptionsToPackage{dvipsnames,svgnames*,x11names*}{xcolor}
%
\documentclass[
  8pt,
  ignorenonframetext,
  aspectratio=169]{beamer}
\usepackage{pgfpages}
\setbeamertemplate{caption}[numbered]
\setbeamertemplate{caption label separator}{: }
\setbeamercolor{caption name}{fg=normal text.fg}
\beamertemplatenavigationsymbolsempty
% Prevent slide breaks in the middle of a paragraph
\widowpenalties 1 10000
\raggedbottom
\setbeamertemplate{part page}{
  \centering
  \begin{beamercolorbox}[sep=16pt,center]{part title}
    \usebeamerfont{part title}\insertpart\par
  \end{beamercolorbox}
}
\setbeamertemplate{section page}{
  \centering
  \begin{beamercolorbox}[sep=12pt,center]{part title}
    \usebeamerfont{section title}\insertsection\par
  \end{beamercolorbox}
}
\setbeamertemplate{subsection page}{
  \centering
  \begin{beamercolorbox}[sep=8pt,center]{part title}
    \usebeamerfont{subsection title}\insertsubsection\par
  \end{beamercolorbox}
}
\AtBeginPart{
  \frame{\partpage}
}
\AtBeginSection{
  \ifbibliography
  \else
    \frame{\sectionpage}
  \fi
}
\AtBeginSubsection{
  \frame{\subsectionpage}
}
\usepackage{lmodern}
\usepackage{amssymb,amsmath}
\usepackage{ifxetex,ifluatex}
\ifnum 0\ifxetex 1\fi\ifluatex 1\fi=0 % if pdftex
  \usepackage[T1]{fontenc}
  \usepackage[utf8]{inputenc}
  \usepackage{textcomp} % provide euro and other symbols
\else % if luatex or xetex
  \usepackage{unicode-math}
  \defaultfontfeatures{Scale=MatchLowercase}
  \defaultfontfeatures[\rmfamily]{Ligatures=TeX,Scale=1}
  \setmainfont[]{Microsoft YaHei}
  \setsansfont[]{Microsoft YaHei}
\fi
\usefonttheme{serif} % use mainfont rather than sansfont for slide text
% Use upquote if available, for straight quotes in verbatim environments
\IfFileExists{upquote.sty}{\usepackage{upquote}}{}
\IfFileExists{microtype.sty}{% use microtype if available
  \usepackage[]{microtype}
  \UseMicrotypeSet[protrusion]{basicmath} % disable protrusion for tt fonts
}{}
\makeatletter
\@ifundefined{KOMAClassName}{% if non-KOMA class
  \IfFileExists{parskip.sty}{%
    \usepackage{parskip}
  }{% else
    \setlength{\parindent}{0pt}
    \setlength{\parskip}{6pt plus 2pt minus 1pt}}
}{% if KOMA class
  \KOMAoptions{parskip=half}}
\makeatother
\usepackage{xcolor}
\IfFileExists{xurl.sty}{\usepackage{xurl}}{} % add URL line breaks if available
\IfFileExists{bookmark.sty}{\usepackage{bookmark}}{\usepackage{hyperref}}
\hypersetup{
  pdftitle={LLVM的JIT辅助库ORC JIT介绍},
  pdfauthor={樊其轩},
  colorlinks=true,
  linkcolor=blue,
  filecolor=Maroon,
  citecolor=Blue,
  urlcolor=Blue,
  pdfcreator={LaTeX via pandoc}}
\urlstyle{same} % disable monospaced font for URLs
\newif\ifbibliography
% Make links footnotes instead of hotlinks:
\DeclareRobustCommand{\href}[2]{#2\footnote{\url{#1}}}
\setlength{\emergencystretch}{3em} % prevent overfull lines
\providecommand{\tightlist}{%
  \setlength{\itemsep}{0pt}\setlength{\parskip}{0pt}}
\setcounter{secnumdepth}{-\maxdimen} % remove section numbering
\usecolortheme[light]{solarized}
\useinnertheme{rectangles}
\useoutertheme{infolines}
\usepackage{listings}

% use darkness that is inverse to main solarized theme
\lstset{
    % How/what to match
    sensitive=true,
    % Border (above and below)
    frame=lines,
    % Extra margin on line (align with paragraph)
    xleftmargin=\parindent,
    % Put extra space under caption
    belowcaptionskip=1\baselineskip,
    % Colors
    backgroundcolor=\color{solarizedRebase3},
    basicstyle=\color{solarizedRebase0}\ttfamily,
    keywordstyle=\color{solarizedCyan},
    commentstyle=\color{solarizedRebase01},
    stringstyle=\color{solarizedBlue},
    numberstyle=\color{solarizedViolet},
    identifierstyle=\color{solarizedRebase0},
    %identifierstyle=
    % Break long lines into multiple lines?
    breaklines=true,
    % Show a character for spaces?
    showstringspaces=false,
    tabsize=2
}


\title{LLVM的JIT辅助库ORC JIT介绍}
\author{樊其轩}
\date{\today}

\begin{document}
\frame{\titlepage}

\begin{frame}{ORC是什么,不是什么}
\protect\hypertarget{orcux662fux4ec0ux4e48ux4e0dux662fux4ec0ux4e48}{}
\begin{block}{}
\protect\hypertarget{section}{}
\begin{itemize}
\tightlist
\item
  是辅助开发者实现JIT功能的通用库和运行时,提供一系列JIT用到的支持组件
\item
  可以将LLVM IR动态编译到机器码,但不局限于LLVM IR
\item
  黑箱:交给你代码,查询某个符号,给我使这个符号可用并返回地址
\end{itemize}
\end{block}

\begin{block}{}
\protect\hypertarget{section-1}{}
\begin{itemize}
\tightlist
\item
  不是一个完整的JIT编译器
\item
  不提供动态编译相关的代码分析和优化
\end{itemize}
\end{block}
\end{frame}

\begin{frame}{ORC的功能}
\protect\hypertarget{orcux7684ux529fux80fd}{}
\begin{itemize}
\tightlist
\item
  即时编译,查找一个符号时立刻编译依赖代码
\item
  懒惰编译,延后到某函数符号被调用时再编译依赖代码
\item
  多线程编译和运行
\item
  远端运行,编译和运行分开在不同进程或网络上的不同机器
\item
  像普通程序一样的运行时环境,

  \begin{itemize}
  \tightlist
  \item
    异常处理 (C++的try catch)
  \item
    静态数据初始化(静态全局变量/类初始化)
  \item
    dlopen
  \item
    线程局部存储 (TLS)
  \end{itemize}
\item
  像普通链接器一样的功能,链接顺序,链接静态库和动态库,符号解析
\item
  符号注入,符号动态重定向与重命名,代码动态加载和卸载,缓存编译完成的代码
\item
  支持调试动态生成代码
\item
  提供相对稳定的C ABI,但很多功能都没暴露出来
\item
  自动内存管理,丢弃不需要的中间产物
\end{itemize}
\end{frame}

\begin{frame}[fragile]{ORC的当前局限}
\protect\hypertarget{orcux7684ux5f53ux524dux5c40ux9650}{}
\begin{itemize}
\tightlist
\item
  在多线程环境下提供JIT编译好的代码镜像与相关调试信息

  \begin{itemize}
  \tightlist
  \item
    由于多线程编译以及代码加载与卸载,编译出的代码镜像会发生变动
  \item
    JIT专用链接器也会做死代码删除
  \end{itemize}
\item
  造成的局限

  \begin{itemize}
  \tightlist
  \item
    JIT编译的程序的调试支持是通过一个临时的\texttt{notifyMaterializing}接口实现的
    \footnote<.->{\url{https://github.com/llvm/llvm-project/commit/ef2389235c5dec03be93f8c9585cd9416767ef4c}}

    \begin{itemize}
    \tightlist
    \item
      因为GDB JIT 接口\footnote<.->{\url{https://sourceware.org/gdb/onlinedocs/gdb/JIT-Interface.html}}需要编译好的镜像
    \end{itemize}
  \item
    难以获取行号等等代码附带的信息,除非暂且用\texttt{notifyMaterializing},参考
    Julia\footnote<.->{\url{https://github.com/JuliaLang/julia/blob/master/src/jitlayers.cpp}}

    \begin{itemize}
    \tightlist
    \item
      Linux Perf的JIT接口需要
    \end{itemize}
  \end{itemize}
\item
  未来会研究出更好的机制,来暴露出这方面信息
\end{itemize}
\end{frame}

\begin{frame}[fragile]{ORC的组件}
\protect\hypertarget{orcux7684ux7ec4ux4ef6}{}
\begin{block}{依照抽象层级}
\protect\hypertarget{ux4f9dux7167ux62bdux8c61ux5c42ux7ea7}{}
\begin{itemize}
\tightlist
\item
  JITLink \& RuntimeDyld: JIT链接器,解析符号,链接JIT代码到当前进程

  \begin{itemize}
  \tightlist
  \item
    与其他组件的接口:\texttt{ObjectLinkingLayer}和\texttt{JITLinkContext}
  \item
    Section, Block, Edge, Symbol
  \end{itemize}
\item
  ORC核心: 提供ORC库的基础抽象,各个功能特性都围绕于它
\item
  LLJIT/LazyLLJIT: 方便用户的顶层API,把ORC所有功能整合起来提供给用户

  \begin{itemize}
  \tightlist
  \item
    层级设计
  \item
    \texttt{IRTransformLayer} -\textgreater{} \texttt{IRCompileLayer}
    -\textgreater{} \texttt{ObjectTransformLayer} -\textgreater{}
    \texttt{ObjectLinkingLayer}
  \end{itemize}
\end{itemize}
\end{block}

\begin{block}{依照功能特性}
\protect\hypertarget{ux4f9dux7167ux529fux80fdux7279ux6027}{}
\begin{itemize}
\tightlist
\item
  ExecutorProcessControl(EPC): 负责JIT生成的二进制的运行
\item
  *Platform:用于支持不同平台加载动态共享对象(DSO),compiler-rt下也有相关代码
\item
  DynamicLibrarySearchGenerator: 加载当前进程或者某个动态库的各种符号
\item
  SPS*:
  简单序列化库,在EPC中用来支持与远端通信,调用远端代码,发送编译好的代码
\item
  DebuggerSupportPlugin: JITLink链接器的插件,用来支持JIT程序调试
\item
  JITLinkMemoryManager: JITLink的内存分配器
\item
  等等等
\end{itemize}
\end{block}
\end{frame}

\begin{frame}[fragile]{ORC的组件}
\protect\hypertarget{orcux7684ux7ec4ux4ef6-1}{}
\begin{block}{ORC的核心抽象}
\protect\hypertarget{orcux7684ux6838ux5fc3ux62bdux8c61}{}
\begin{itemize}
\tightlist
\item
  JITDylib

  \begin{itemize}
  \tightlist
  \item
    就像动态库一样,可以连成一串链接(DFS)
  \item
    灵活性,里面可以是任何内容,只要最终能给JIT链接器提供符号和地址
  \item
    懒惰性,里面的代码不需要立刻被编译,不需要立即可用
  \end{itemize}
\item
  ThreadSafeModule和ThreadSafeContext

  \begin{itemize}
  \tightlist
  \item
    LLVMModule和LLVMContext的多线程安全封装
  \item
    \texttt{WithModuleDo}
  \end{itemize}
\item
  MaterializationUnit

  \begin{itemize}
  \tightlist
  \item
    记录可以一块被编译并解析好 (materialize)的符号
  \item
    \texttt{materialize()}和\texttt{discard()}接口定义如何materialize这些符号
  \item
    AbsoluteMaterializationUnit
  \end{itemize}
\item
  MaterializationResponsibility

  \begin{itemize}
  \tightlist
  \item
    用于多线程环境下传递和追踪错误信息
  \item
    \texttt{notifyResolved()}, \texttt{notifyEmitted()},
    \texttt{failMaterialization()}
  \end{itemize}
\item
  ResourceTracker

  \begin{itemize}
  \tightlist
  \item
    用来支持代码加载和卸载
  \item
    一般每个JITDylib会有一个,也可以定义多个
  \end{itemize}
\item
  ExecutionSession

  \begin{itemize}
  \tightlist
  \item
    代表一个JIT运行实例,包含所有JITDylib,符号名称字符串池,EPC等等
  \item
    带锁,协调JIT全局各个组件,向相应JITDylib分发符号查询和编译请求
  \end{itemize}
\end{itemize}
\end{block}
\end{frame}

\begin{frame}[fragile]{ORC使用}
\protect\hypertarget{orcux4f7fux7528}{}
文档里的简单示例代码,省去错误处理

\begin{lstlisting}[language=c++,]
auto JIT = LLJITBuilder().create();
// Add the module.
JIT->addIRModule(TheadSafeModule(std::move(M), Ctx));

// Look up the JIT'd code entry point.
auto EntrySym = JIT->lookup("entry");
if (!EntrySym)
  return EntrySym.takeError();

// Cast the entry point address to a function pointer.
auto *Entry = EntrySym.getAddress().toPtr<void(*)()>();
// Call into JIT'd code.
Entry();
\end{lstlisting}

LLJIT的更多例子
\url{https://github.com/llvm/llvm-project/tree/main/llvm/examples/OrcV2Examples}
\end{frame}

\begin{frame}[fragile]{ORC与MCJIT对比}
\protect\hypertarget{orcux4e0emcjitux5bf9ux6bd4}{}
\begin{columns}[T]
\begin{column}{0.48\textwidth}
\begin{block}{MCJIT}
\protect\hypertarget{mcjit}{}
\begin{itemize}
\tightlist
\item
  对应接口很繁琐,参看ExecutionEngine类\footnote<.->{\url{https://llvm.org/doxygen/classllvm_1_1ExecutionEngine.html}}有多少
  \texttt{get\textbackslash{}*Address},\texttt{getPointerTo\textbackslash{}*}这种函数
\item
  垒代码山垒出来的接口 (adhoc),有设计问题
\item
  一个MCJIT实例单独处理一个JIT代码模块
\item
  复用LLVM + JIT专门的RuntimeDyld链接器
\item
  提供基础JIT功能,缺少一些功能如small
  codemodel支持,部分功能不完善不稳定比如TLS和Windows COFF格式
\item
  整个JIT流程简单固定,缺少对内部流程的可拓展性
\item
  没有的自动内存管理,内存浪费
\end{itemize}
\end{block}
\end{column}

\begin{column}{0.48\textwidth}
\begin{block}{ORC}
\protect\hypertarget{orc}{}
\begin{itemize}
\tightlist
\item
  提供方便的接口,来查找JIT出的函数/符号地址
\item
  一个LLJIT实例(单例模式)处理多个JIT代码模块
\item
  同时支持RuntimeDyld和新的灵活的JITLink链接器
\item
  支持更丰富更完整的功能
\item
  内部流程都可定制,灵活可拓展的接口,且各个组件相对独立可单独使用
\item
  提供自动内存管理,并可定制
\end{itemize}
\end{block}
\end{column}
\end{columns}
\end{frame}

\begin{frame}{ORC实际应用}
\protect\hypertarget{orcux5b9eux9645ux5e94ux7528}{}
\begin{block}{}
\protect\hypertarget{section-2}{}
\begin{itemize}
\tightlist
\item
  Julia编译器\footnote<.->{\url{https://github.com/JuliaLang/julia/blob/master/src/jitlayers.cpp}}
  用到了很多复杂功能
\item
  CLASP 基于LLVM实现的common lisp编译器\footnote<.->{\url{https://github.com/clasp-developers/clasp}}
\end{itemize}
\end{block}

\begin{block}{}
\protect\hypertarget{section-3}{}
\begin{itemize}
\tightlist
\item
  PostgreSQL 重复且昂贵的查询指令JIT到二进制来加速
\item
  Numba (通过llvmlite) \footnote<.->{\url{https://github.com/numba/llvmlite/pull/942}}
\item
  LLVM IR的解释器 lli
\end{itemize}
\end{block}

\begin{block}{}
\protect\hypertarget{section-4}{}
\begin{itemize}
\tightlist
\item
  LLVM BOLT 单独使用了底层的JITLink链接器\footnote<.->{\url{https://github.com/llvm/llvm-project/commit/05634f7346a59f6dab89cde53f39b40d9a70b9c9}}
\item
  把C++当脚本语言写,Facebook的用例\footnote<.->{\url{https://www.youtube.com/watch?v=01WoFnyw6zE}}
\item
  Clang-Repl 交互式C++解释器\footnote<.->{\url{https://clang.llvm.org/docs/ClangRepl.html}}
\end{itemize}
\end{block}
\end{frame}

\begin{frame}[fragile]{相关资料推荐}
\protect\hypertarget{ux76f8ux5173ux8d44ux6599ux63a8ux8350}{}
LLVM官方文档中的Kaleidoscope教程有实操用ORC给编译器添加JIT功能章节,很推荐

除去ORC官方文档\footnote<.->{注意,由于ORC发展挺快,官方文档有一定滞后}还有:

\begin{itemize}
\tightlist
\item
  Lang Hames的Youtube上ORC介绍视频,一搜就有总共4个
\item
  Sunhao的JITLINK介绍和windows支持

  \begin{itemize}
  \tightlist
  \item
    \url{https://www.youtube.com/watch?v=UwHgCqQ2DDA}
  \end{itemize}
\item
  LLVM Discord服务器上的\texttt{\#jit}频道,开发者在此非常活跃
\item
  我之前的LLVM ORC JIT 简介\footnote<.->{有错误,欢迎指正。内容远不及看上面的材料和动手实验}

  \begin{itemize}
  \tightlist
  \item
    \url{https://www.bilibili.com/video/BV13a41187NM}
  \end{itemize}
\end{itemize}
\end{frame}

\end{document}
